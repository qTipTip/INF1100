\documentclass{article}

\usepackage[]{minted} 

\begin{document}
  \section{Python properties and function decorators}
  \label{sec:python_properties_and_function_decorators}
  
  In programming we often speak of \emph{private} and \emph{public} class
  attributes. Public attributes are those which anyone outside of the class can
  make use of, where as private attributes are those attributes which should be
  hidden from the user. One typically implements what we call \emph{getter} and
  \emph{setter} methods for retrieving and assigning values, respectively, to
  private variables.

  A classical example that can help emphasize the difference between public and
  private attributes is a Python class representing a bank account. Associated
  with a bank-account is amongst other an \emph{account number}, an
  \emph{account balance} and methods for depositing and retrieving money. A
  typical implementation might look like:
  \begin{minted}{python}
class Account:
  def __init__(self, name, account_number, initial_amount):
    self.name = name
    self.no = account_number
    self.balance = initial_amount
  
  def deposit(self, amount):
    self.balance += amount

  def withdraw(self, amount):
    self.balance -= amount

  def dump(self):
    print '%s, %s, balance: %s' % (self.name, self.no, self.balance)
  \end{minted}
  We can now create an instance of this account class and use its methods as follows:
  \begin{minted}{python}
test_account = Account('Ola Nordmann', '123456789', 20000) 

test_account.deposit(1000)
test_account.dump()
test_account.withdraw(15000)
test_account.dump()
  \end{minted}
  However, there is nothing stopping a user from writing the following code!
  \begin{minted}{python}
test_account.balance = 10000000 
  \end{minted}
\end{document}
